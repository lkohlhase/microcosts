\documentclass{article}
\usepackage{listings}
\usepackage{color}
\usepackage{amsmath}

\definecolor{dkgreen}{rgb}{0,0.6,0}
\definecolor{gray}{rgb}{0.5,0.5,0.5}
\definecolor{mauve}{rgb}{0.58,0,0.82}

\lstset{frame=tb,
  language=C++,
  aboveskip=3mm,
  belowskip=3mm,
  showstringspaces=false,
  columns=flexible,
  basicstyle={\small\ttfamily},
  numbers=none,
  numberstyle=\tiny\color{gray},
  keywordstyle=\color{blue},
  commentstyle=\color{dkgreen},
  stringstyle=\color{mauve},
  breaklines=true,
  breakatwhitespace=true,
  tabsize=3
}
\title{Practice Breaking down of QRSdet algorithm}
\author{Lukas Kohlhase}
\begin{document}
\maketitle
We will look at the individual sub functions, and then the main one 
\section{Filters}
The main filter function: 
\begin{lstlisting}
int QRSFilter(int datum,int init)
	{
	int fdatum ;

	if(init)
		{
		hpfilt( 0, 1 ) ;		// Initialize filters.
		lpfilt( 0, 1 ) ;
		mvwint( 0, 1 ) ;
		deriv1( 0, 1 ) ;
		deriv2( 0, 1 ) ;
		}

	fdatum = lpfilt( datum, 0 ) ;		// Low pass filter data.
	fdatum = hpfilt( fdatum, 0 ) ;	// High pass filter data.
	fdatum = deriv2( fdatum, 0 ) ;	// Take the derivative.
	fdatum = abs(fdatum) ;				// Take the absolute value.
	fdatum = mvwint( fdatum, 0 ) ;	// Average over an 80 ms window .
	return(fdatum) ;
	}
\end{lstlisting}
Basically just takes the separate ones listing functions
\subsection{lpfilt}
A low pass filter based on the equation $y[n] = 2*y[n-1] - y[n-2] + x[n] - 2*x[t-24 ms] + x[t-48 ms]$. \\ 
Two additions for bookkeeping purposes. 

Main equation is two left shifts and three additions.
\subsection{hpfilt}
A high pass filter based on the equations
\begin{align*}
	y[n] &= y[n-1] + x[n] - x[n-128 ms] \\
  z[n] &= x[n-64 ms] - y[n] ;
\end{align*}
Two additions for bookkeeping purposes 

Main two equations are three additions and one division.

\subsection{deriv1}
It's the easy way to do derivation, just $x[n]-x[n-10ms]$, so we have just on addition. Deriv2 is the same way
\subsection{mvwint}
Just averaging over a moving window. 

One addition for administrative stuff, 2 for averaging and one division

\subsection{In total}
Twelve additions, two left shifts and two divisions
\section{Main Algorithm}
We will first do the subalgorithms, and then the main algorithm 
\subsection{Peak}
It has one addition, a bunch of comparisons and variable reassignments.
\end{document}
